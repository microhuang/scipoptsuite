\documentclass[11pt,listof=totoc]{scrartcl}
\usepackage[utf8x]{inputenc}
\usepackage[T1]{fontenc}
\usepackage{amsmath, amsfonts, amssymb, amsthm}
\usepackage[colorlinks]{hyperref}
\usepackage[capitalize]{cleveref}
\usepackage{url}
\usepackage[greek,english]{babel}

\theoremstyle{definition}
\newtheorem{ex}{Example}[section]

\title{PolySCIP user guide}
\author{Sebastian Schenker}
\date{}

\begin{document}
\maketitle
\tableofcontents

\section{General information}
PolySCIP is a solver for multi-criteria integer programming as well as
multi-criteria linear programming with an arbitrary number of
objectives. In other words, it solves optimization problems of the
form:
\begin{align*}
\min / \max~ &(c_1^T x, \ldots, c_k^T x) \\
\mbox{s.t. } Ax &\leq b,\\
x &\in \mathbb{Z}^n \lor \mathbb{Q}^n.
\end{align*}
where $k \geq 2,~ A \in \mathbb{Q}^{m \times n},~ b \in
\mathbb{Q}^m$.

The name PolySCIP is composed of Poly (from the Greek
\textgreek{pol'us} meaning ``many'') and SCIP. PolySCIP is part of \href{http://scip.zib.de}{SCIP} and its source code resides in the 'applications' directory.

\section{Installation}

See the INSTALL file in the PolySCIP directory or the section 'Installation' at \url{http://polyscip.zib.de}.

\section{Usage}

The problem file (in \texttt{MOP} format) is the only mandatory command line argument:
\begin{verbatim}
./polyscip /path_to_problem_file/problem_file.mop
\end{verbatim} would run PolySCIP on the given file. To switch off the SCIP solver output you can execute 
\begin{verbatim}
./polyscip -p scipmip.set /path_to_problem_file/problem_file.mop
\end{verbatim} where \texttt{scipmip.set} is a parameter file with SCIP Parameters 
included in the PolySCIP directory. For more details about SCIP
parameters, see the command line argument \textbf{-p} described below.

Command line arguments are:
\begin{description}
\item[-h, -{}-help] Displays usage information and exits
\item[-v, -{}-verbose] Switches on verbose PolySCIP output
\item[-t <sec>, -{}-timeLimit <sec>] Sets a time limit in seconds on the overall computation time
\item[ -p <\texttt{param\_file.set}>, -{}-paramSets <\texttt{param\_file.set}>] Specifies a file consisting of SCIP parameter settings
\begin{itemize}
 \item PolySCIP comes with the parameter settings file \emph{scipmip.set}
 \item a \href{http://scip.zib.de/doc/html_devel/PARAMETERS.php}{list} of all available SCIP parameters
 \item e.g., to switch the verbosity level of the internal SCIP solution process to 1, write \texttt{display/verblevel=1} in the \emph{scipmip.set} file and run \emph{polyscip} with \texttt{-p scipmip.set}
\end{itemize}
\item[-w, -{}-writeSols] Write solutions to a file; default path is ./
\item[-W <path>, -{}-writeSolsPath <path>] Path where the solution file should be written to if \textbf{-w} was set
\item[-{}-version] Displays version information and exits
\end{description}

\section{File format}\label{sec-format}

The PolySCIP file format (with suffix \texttt{.mop}) is based on the
widely used \texttt{MPS} file format (see \cite{mps-format},
\cite{mps-format2}). \texttt{MPS} is column-oriented and all model
components (variables, rows, etc.)  receive a name. An objective in
\texttt{MPS} is indicated by an \texttt{N} followed by the name in the
\texttt{ROWS} section. Similarly, in the \texttt{MOP} format the
objectives are indicated by \texttt{N} followed by the name in the
\texttt{ROWS} section. In general, \texttt{MPS} might not be as human
readable as other formats. However, one of the main reasons to base
the file format of PolySCIP on it is its easy extension towards
several objectives and its wide availability in most of the linear and
integer programming software packages such that available \texttt{MPS}
parsers could easily be adjusted to parse an \texttt{.mop} file as
well. Furthermore, no user is expected to write \texttt{.mop} files by
hand, but to use a modelling language that does the job. See
\cref{sec-model} for a description of how to use the freely available
\href{http://zimpl.zib.de}{Zimpl} and the Python script
\texttt{mult\_zimpl\_to\_mop.py} (comes with PolySCIP) to generate
\texttt{.mop} files.

The following simple equation-based bi-criteria
integer problem

\begin{alignat*}{4}
&\mbox{maximize}~~~ &&\mbox{Obj1: } 3&&x_1 + 2 x_2 - 4 x_3 &&\\
& &&\mbox{Obj2: } &&x_1 + x_2 + 2 x_3 &&\\
&\mbox{subject to} && && && \\
& &&\mbox{Eqn: } &&x_1 + x_2 + x_3 &&= 2 \\
& &&\mbox{Lower: } &&x_1 + 0.4 x_2 &&\leq 1.5 \\
& && &&x_1,~ x_2,~ x_3 &&\geq 0 \\
& && &&x_1,~ x_2,~ x_3 &&\in \mathbb{Z}
\end{alignat*}
is written in MOP format as follows:
\begin{verbatim}
NAME        BICRIT
OBJSENSE 
 MAX
ROWS
 N  Obj1            
 N  Obj2            
 E  Eqn             
 L  Lower           
COLUMNS
    x#1       Lower              1
    x#1       Eqn                1
    x#1       Obj2               1
    x#1       Obj1               3
    x#2       Lower            0.4
    x#2       Eqn                1
    x#2       Obj2               1
    x#2       Obj1               2
    x#3       Eqn                1
    x#3       Obj2               2
    x#3       Obj1              -4
RHS
    RHS       Eqn                2
    RHS       Lower            1.5
BOUNDS
 LI BOUND     x#1                0
 LI BOUND     x#2                0
 LI BOUND     x#3                0
ENDATA
\end{verbatim}

\section{User-friendly \texttt{.mop} file generation}\label{sec-model}

\href{http://zimpl.zib.de}{Zimpl} is a freely available modelling
language (also part of the SCIP Optimization Suite) to translate a
mathematical model of a problem into a mathematical program in
\texttt{.mps} (or \texttt{.lp}) file format. Together with the
\texttt{mult\_zimpl\_to\_mop.py} script (located in the 'mult\_zimpl'
directory of PolySCIP) it can/should be used to generate your
\texttt{.mop} files. For a more detailed description of Zimpl, see the
Zimpl \href{http://zimpl.zib.de/download/zimpl.pdf}{user guide}. In
this section we will only describe how to make use of it, but not all
options how to write different models. Zimpl does generally not support several
objectives; this is where \texttt{mult\_zimpl\_to\_mop.py} comes into
play. It takes an 'extended' Zimpl file containing several objectives,
internally rewrites all but the first objectives into constraints,
calls Zimpl on the rewritten file and changes the file generated by Zimpl
containing 'artificial' constraint indicators back to objective
indicators which yields an \texttt{.mop} file.

\begin{itemize}
\item Zimpl comes with the \href{http://scip.zib.de/#scipoptsuite}{SCIP Optimization Suite}
\begin{itemize}
\item Please see the INSTALL file of the SCIP Optimization Suite (you
  basically just need the GMP library in order to build).
\end{itemize}
\item \texttt{mult\_zimpl\_to\_mop.py} is a Python3 script and comes with PolySCIP; it is located in the 'mult\_zimpl' directory
\begin{itemize}
\item Execute \texttt{python3 mult\_zimpl\_to\_mop.py your\_model.zpl} to run it on the file \texttt{your\_model.zpl} containing your multi-criteria model
\item The following command line arguments are available
\begin{description}
\item[-h, -{}-help] Shows the help message and exits
\item[-p <path>] The directory where the generated \texttt{.mop} file should be saved
\item[--path\_to\_zimpl <path>] The directory where your \emph{zimpl} binary is located
\end{description}
\end{itemize}
\item[] E.g., if Zimpl was not installed globally but in
  \texttt{/home/user/zimpl}, and, furthermore, if you would like to save
  the generated \texttt{.mop} file in \texttt{/tmp/}, then run
  \texttt{python3 mult\_zimpl\_to\_mop.py -p /tmp/ --path\_to\_zimpl
    /home/user/zimpl/bin your\_model.zpl}
\end{itemize}

Please note (in the following examples) that the direction of optimization,
i.e., \texttt{minimize} or \texttt{maximize}, is declared only once
followed by the first objective. All other objectives follow without a
direction specification implying that all objectives are assumed to be
either minimized or maximized.

\begin{ex}
The bi-criteria maximization problem of \cref{sec-format} could be modelled as follows:
\begin{verbatim}
set I := {1..3};
param c1[I] := <1> 3, <2> 2, <3> -4;     #coefficients of the first objective
param c2[I] := <3> 2 default 1;          #coefficients of the second objective 
param low[I] := <1> 1, <2> 0.4, <3> 0;   #coefficients of the lower constraint
var x[I] integer >= 0;

maximize Obj1: sum <i> in I: c1[i]*x[i];
Obj2: sum <i> in I: c2[i]*x[i];

subto Eqn: sum <i> in I: x[i] == 2;
subto Lower: sum <i> in I: low[i]*x[i] <= 1.5;
\end{verbatim}
Saving the file, e.g., as test.zpl and running \texttt{mult\_zimpl\_to\_mop.py} on it would generate a file named \texttt{test.mop} which can be solved with PolySCIP. 
\end{ex}

\begin{ex}A tri-criteria linear programming minimization problem with a four-dimensional standard cube as feasible space and three unit vectors as objectives could be modelled as follows:
\begin{verbatim}
set I := {1..4};
var x[I] real;
minimize Obj1: x[1];
Obj2: x[2];
Obj3: x[3];
subto Bounds: forall <i> in I: 0 <= x[i] <= 1;
\end{verbatim}
Again, saving the file, e.g., as testCube.zpl and running \texttt{mult\_zimpl\_to\_mop.py} on it would generate a file named \texttt{testCube.mop} which can be solved with PolySCIP. 
\end{ex}

For more involved models there is also the possibility to read
external files containing data for parameters. Please the see the
Zimpl \href{http://zimpl.zib.de/download/zimpl.pdf}{user guide} for
more details.

\begin{thebibliography}{}
\bibitem{mps-format} MPS format (short), 
  \url{https://en.wikipedia.org/wiki/MPS_%28format%29}

\bibitem{mps-format2} MPS format (detailed),
  \url{http://lpsolve.sourceforge.net/5.5/mps-format.htm}
\end{thebibliography}

\end{document}